\documentclass[]{article}

\usepackage{float}
\usepackage{graphicx}

\usepackage{titling}
\newcommand{\subtitle}[1]{%
  \posttitle{%
    \par\end{center}
    \begin{center}\large#1\end{center}
    \vskip0.5em}%
}

\begin{document}

\title{Lab 1}
\subtitle{CS M152A}
\author{Aman Agarwal \& Lowell Bander}

\maketitle

\section{FPGA Design Workflow}
\section{Example Program}

\begin{table}[H]
\centering
\begin{tabular}{ l | l }
\textbf{Sequencer Instruction} & \textbf{Binary}\\\hline
\texttt{PUSH R0 0x4} & \texttt{0000 0010}\\
\texttt{PUSH R0 0x0} & \texttt{0000 0000}\\
\texttt{PUSH R1 0x3} & \texttt{0001 0011}\\
\texttt{MULT R0 R1 R2} & \texttt{1000 0110}\\
\texttt{ADD R2 R0 R3} & \texttt{0110 0011}\\
\texttt{PUSH R0 0x4} & \texttt{0000 0100}\\
\texttt{SEND R0} & \texttt{1100  xxxx}\\
\texttt{SEND R1} & \texttt{1101  xxxx}\\
\texttt{SEND R2} & \texttt{1110  xxxx}\\
\texttt{SEND R3} & \texttt{1111  xxxx}\\
\end{tabular}
\caption{Sequencer instructions translated to binary. The four least significant bits are `don't care' as their value is not used for the \texttt{SEND} instruction.}
\label{table:translation}
\end{table}

\begin{figure}[H]
\centering
\includegraphics[width=10cm]{translation.png}
\caption{UART console output resulting from the executions in Table~\ref{table:translation}.}
\end{figure}


\section{Fibonacci}
\section{Exercise 1}
\section{Exercise 2}

\end{document}