\documentclass[]{article}

\usepackage{listings}
\usepackage{float}
\usepackage{graphicx}

\usepackage{titling}
\newcommand{\subtitle}[1]{%
  \posttitle{%
    \par\end{center}
    \begin{center}\large#1\end{center}
    \vskip0.5em}%
}

\begin{document}

\title{Lab 3: Stopwatch}
\subtitle{CS M152A}
\author{Aman Agarwal \& Lowell Bander}

\maketitle
\tableofcontents \newpage

\section{Introduction}

In this lab, we implemented a simple stopwatch. The stopwatch counts up from 00:00 to 59:59 over the course of one hour during normal operation, then resets to 00:00. \\

Additionally, the stopwatch can be paused, manually reset, and the minutes and the seconds stopwatch can be modified individually at a rate of twice the normal rate. During such modification, the digits being modified blink to indicate their modification.

\section{Design Description}

\subsection{High Level Design}

\subsection{Low Level Implementation}

\section{Simulation Documentation}

In our first labs, we made the faux pas of writing all of our code before testing any of it. This time, we developed in a very incremental and modular manner. For example, we began with a simple counter module, created a test bench for it, then chained them together to make a larger module, then tested that module, and so on. We hope this made it easy to catch issues early on in the development process.\\

In addition to testing each module as it was developed, we tested the entire device as a whole once development was complete. Namely, we tested every state transition pictured in Figure ???.\\

During simulation, we found several errors. For example, we discovered that the module which divides our clock wasn't functioning properly for one of our clocks.

\section{Conclusion}


\end{document}

